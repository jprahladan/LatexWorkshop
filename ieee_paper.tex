
%% bare_conf.tex
%% V1.3
%% 2007/01/11
%% by Michael Shell
%% See:
%% http://www.michaelshell.org/
%% for current contact information.
%%
%% This is a skeleton file demonstrating the use of IEEEtran.cls
%% (requires IEEEtran.cls version 1.7 or later) with an IEEE conference paper.
%%
%% Support sites:
%% http://www.michaelshell.org/tex/ieeetran/
%% http://www.ctan.org/tex-archive/macros/latex/contrib/IEEEtran/
%% and
%% http://www.ieee.org/

%%*************************************************************************
%% Legal Notice:
%% This code is offered as-is without any warranty either expressed or
%% implied; without even the implied warranty of MERCHANTABILITY or
%% FITNESS FOR A PARTICULAR PURPOSE! 
%% User assumes all risk.
%% In no event shall IEEE or any contributor to this code be liable for
%% any damages or losses, including, but not limited to, incidental,
%% consequential, or any other damages, resulting from the use or misuse
%% of any information contained here.
%%
%% All comments are the opinions of their respective authors and are not
%% necessarily endorsed by the IEEE.
%%
%% This work is distributed under the LaTeX Project Public License (LPPL)
%% ( http://www.latex-project.org/ ) version 1.3, and may be freely used,
%% distributed and modified. A copy of the LPPL, version 1.3, is included
%% in the base LaTeX documentation of all distributions of LaTeX released
%% 2003/12/01 or later.
%% Retain all contribution notices and credits.
%% ** Modified files should be clearly indicated as such, including  **
%% ** renaming them and changing author support contact information. **
%%
%% File list of work: IEEEtran.cls, IEEEtran_HOWTO.pdf, bare_adv.tex,
%%                    bare_conf.tex, bare_jrnl.tex, bare_jrnl_compsoc.tex
%%*************************************************************************

% *** Authors should verify (and, if needed, correct) their LaTeX system  ***
% *** with the testflow diagnostic prior to trusting their LaTeX platform ***
% *** with production work. IEEE's font choices can trigger bugs that do  ***
% *** not appear when using other class files.                            ***
% The testflow support page is at:
% http://www.michaelshell.org/tex/testflow/



% Note that the a4paper option is mainly intended so that authors in
% countries using A4 can easily print to A4 and see how their papers will
% look in print - the typesetting of the document will not typically be
% affected with changes in paper size (but the bottom and side margins will).
% Use the testflow package mentioned above to verify correct handling of
% both paper sizes by the user's LaTeX system.
%
% Also note that the "draftcls" or "draftclsnofoot", not "draft", option
% should be used if it is desired that the figures are to be displayed in
% draft mode.
%
\documentclass[conference]{IEEEtran}

\usepackage{graphicx}
\usepackage{amsfonts}
\usepackage{amsmath} %for equations
\usepackage{mathtools}

% *** Do not adjust lengths that control margins, column widths, etc. ***
% *** Do not use packages that alter fonts (such as pslatex).         ***
% There should be no need to do such things with IEEEtran.cls V1.6 and later.
% (Unless specifically asked to do so by the journal or conference you plan
% to submit to, of course. )

% correct bad hyphenation here
\hyphenation{op-tical net-works semi-conduc-tor}


\begin{document}
%
% paper title
% can use linebreaks \\ within to get better formatting as desired
\title{Safety Verification of Hybrid Systems Using Barrier Certificates}


% author names and affiliations
% use a multiple column layout for up to three different
% affiliations
\author{
%\IEEEauthorblockN{Jacob Cook}
%\IEEEauthorblockA{School of Electrical and Computer Engineering\\
%University of Colorado\\
%Boulder, Colorado 80303\\
%Email: jacob.cook@colorado.edu}
%\and
\IEEEauthorblockN{Prasanth Prahladan}
\IEEEauthorblockA{School of Electrical and Computer Engineering\\
University of Colorado\\
Boulder, Colorado 80303\\
Email: prasanth.prahladan@colorado.edu}}

% make the title area
\maketitle


\begin{abstract}
%\boldmath
In this paper we study and summarize the journal articles "Safety Verification of Hybrid Systems Using Barrier Certificates", by Stephen Prajna and Ali-Jadbabie, and "Semidefinite programming relaxations for semialgebraic problems" by Pablo Parrilo. The primary focus is on understanding how Semidefinite Programming can be used to determine Barrier Certificates for a system, which can be used to Verify Safety properties of Hybrid Systems. 
The author does not make any claims of original work in this paper, only seeks to summarize the results and present the appropriate theory to be more accessible. 
\end{abstract}

\IEEEpeerreviewmaketitle

\section{Introduction: Safety Verification of Hybrid Systems}
The ubiquity of engineering and physical systems that can be modelled as hybrid systems, has motivated much of research effort to the development of hybrid systems theory. Safety verification of these systems which can exhibit complex behaviours from the combination of switched and continuous dynamics, thus becomes extremely critical and challenging. 

Traditionally mature frameworks for safety verification of systems - Temporal Logic(discrete systems) and Robust Control ( continuous systems), are inadequate for handling hybrid systems. Several methods have been proposed for verification of hybrid systems. Most popular methods, involve explicit computation of either exact or approximate reachable sets corresponding to the continuous dynamics of the different modes of the system. 

The paper by Prajna and Jadbabaie \cite{prajna04}, presents a new method for safety verification of hybrid systems that does not require computation of reachable sets, but instead relies on a mathematical structure called Barrier certificates, which are \textit{real functionals} defined over the state space satisfying a set of inequalities on itself, and its time-derivative along the flow of the system. The methodology, works similar to the technique of using Lyapunov functions to prove stability of a system, without explicitly computing the flow/trajectories of the system. The Barrier-Certificates approach is broadly applicable, with extensions that apply to a large class of hybrid systems including those with nonlinear constraints, uncertain inputs/parameters, uncertain constraints and dynamic constraints.
%Barrier certificates, have been previously used to verify continuous systems with non-linear dynamics. The paper discusses how the techniques may be extended to handle the case of Hybrid Systems. 

When the vector fields of a system are polynomials and the sets in tehs ysetm description are semi-algebraic(i.e. described by polynomial (in)equalities), a tractable computaitonal method using the sum of squares decomposition(SOS-D) and semidefinite programming can be utilized for constructing a polynomial Barrier certificate. The computational cost of this construction shall depend on
\begin{itemize}
\item degree of vector fields,
\item degree of barrier certificate ,
\item dimension of the continuous state-space.
\end{itemize}
However, for fixed degree, the \textbf{computational complexity is polynomial} with respect to state-dimension. 

 
\hfill January 11, 2007

\subsection{Subsection Heading Here}
Subsection text here.


\subsubsection{Subsubsection Heading Here}
Subsubsection text here.


% An example of a floating figure using the graphicx package.
% Note that \label must occur AFTER (or within) \caption.
% For figures, \caption should occur after the \includegraphics.
% Note that IEEEtran v1.7 and later has special internal code that
% is designed to preserve the operation of \label within \caption
% even when the captionsoff option is in effect. However, because
% of issues like this, it may be the safest practice to put all your
% \label just after \caption rather than within \caption{}.
%
% Reminder: the "draftcls" or "draftclsnofoot", not "draft", class
% option should be used if it is desired that the figures are to be
% displayed while in draft mode.
%
%\begin{figure}[!t]
%\centering
%\includegraphics[width=2.5in]{myfigure}
% where an .eps filename suffix will be assumed under latex, 
% and a .pdf suffix will be assumed for pdflatex; or what has been declared
% via \DeclareGraphicsExtensions.
%\caption{Simulation Results}
%\label{fig_sim}
%\end{figure}

% Note that IEEE typically puts floats only at the top, even when this
% results in a large percentage of a column being occupied by floats.


% An example of a double column floating figure using two subfigures.
% (The subfig.sty package must be loaded for this to work.)
% The subfigure \label commands are set within each subfloat command, the
% \label for the overall figure must come after \caption.
% \hfil must be used as a separator to get equal spacing.
% The subfigure.sty package works much the same way, except \subfigure is
% used instead of \subfloat.
%
%\begin{figure*}[!t]
%\centerline{\subfloat[Case I]\includegraphics[width=2.5in]{subfigcase1}%
%\label{fig_first_case}}
%\hfil
%\subfloat[Case II]{\includegraphics[width=2.5in]{subfigcase2}%
%\label{fig_second_case}}}
%\caption{Simulation results}
%\label{fig_sim}
%\end{figure*}
%
% Note that often IEEE papers with subfigures do not employ subfigure
% captions (using the optional argument to \subfloat), but instead will
% reference/describe all of them (a), (b), etc., within the main caption.


% An example of a floating table. Note that, for IEEE style tables, the 
% \caption command should come BEFORE the table. Table text will default to
% \footnotesize as IEEE normally uses this smaller font for tables.
% The \label must come after \caption as always.
%
%\begin{table}[!t]
%% increase table row spacing, adjust to taste
%\renewcommand{\arraystretch}{1.3}
% if using array.sty, it might be a good idea to tweak the value of
% \extrarowheight as needed to properly center the text within the cells
%\caption{An Example of a Table}
%\label{table_example}
%\centering
%% Some packages, such as MDW tools, offer better commands for making tables
%% than the plain LaTeX2e tabular which is used here.
%\begin{tabular}{|c||c|}
%\hline
%One & Two\\
%\hline
%Three & Four\\
%\hline
%\end{tabular}
%\end{table}


% Note that IEEE does not put floats in the very first column - or typically
% anywhere on the first page for that matter. Also, in-text middle ("here")
% positioning is not used. Most IEEE journals/conferences use top floats
% exclusively. Note that, LaTeX2e, unlike IEEE journals/conferences, places
% footnotes above bottom floats. This can be corrected via the \fnbelowfloat
% command of the stfloats package.



\section{Conclusion}
The conclusion goes here.




% conference papers do not normally have an appendix


% use section* for acknowledgement
\section*{Acknowledgment}


The authors would like to thank...





% trigger a \newpage just before the given reference
% number - used to balance the columns on the last page
% adjust value as needed - may need to be readjusted if
% the document is modified later
%\IEEEtriggeratref{8}
% The "triggered" command can be changed if desired:
%\IEEEtriggercmd{\enlargethispage{-5in}}

% references section

% can use a bibliography generated by BibTeX as a .bbl file
% BibTeX documentation can be easily obtained at:
% http://www.ctan.org/tex-archive/biblio/bibtex/contrib/doc/
% The IEEEtran BibTeX style support page is at:
% http://www.michaelshell.org/tex/ieeetran/bibtex/
%\bibliographystyle{IEEEtran}
% argument is your BibTeX string definitions and bibliography database(s)
%\bibliography{IEEEabrv,../bib/paper}
%
% <OR> manually copy in the resultant .bbl file
% set second argument of \begin to the number of references
% (used to reserve space for the reference number labels box)
\begin{thebibliography}{1}

\bibitem{prajna04}
Prajna, Stephen, and Ali Jadbabaie. \textit{"Safety verification of hybrid systems using barrier certificates."} Hybrid Systems: Computation and Control. Springer Berlin Heidelberg, 2004. 477-492.
\bibitem{pablo03}
Parrilo, Pablo A. \textit{"Semidefinite programming relaxations for semialgebraic problems."} Mathematical programming 96.2 (2003): 293-320.


\end{thebibliography}




% that's all folks
\end{document}


